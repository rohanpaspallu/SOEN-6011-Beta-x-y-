\documentclass{article}
\usepackage[utf8]{inputenc}
\usepackage{fullpage}
\usepackage{enumitem}



% \author{ROHAN DEEPAK PASPALLU}
% \date{July 2019}

\begin{document}
\noindent
\Large\textbf{PROBLEM-1}\hfill\textbf{ROHAN DEEPAK PASPALLU}\\
\textbf{Topic: $B(x,y)$}\hfill\textbf{40093648}

\section{Introduction}
    $B(x,y)$ denotes the beta function also known as the Euler’s beta function.
    Beta function is a function which is defined for the values defined in a certain specific limits of a function.
    The formula of \textbf{$B(x,y)$} is:
    \begin{itemize}
        \item $B(x,y) = \frac{(x-1)! (y-1)!}{(x+y-1)!}$ \textbf{For positive integers}\cite{wiki}
        \item $B(x,y)$ = $\int_{0}^{1} \frac{t^{x-1}}{(1-t)^{y-1}} dt$ \textbf{For positive real numbers}
    \end{itemize}

\section{Properties of Beta Function}
    \begin{itemize}[noitemsep]
        \item Beta function is symmetric: $B(x,y) = B(y,x)$
        \item Beta function in terms of Gamma functions as:
            $B(x,y)=\frac{\Gamma x \Gamma y}{\Gamma (x+y)}$
        \item When x and y are postitive then it follows the form of gamma function.
        \item There can me multiple parameters in the beta function (i.e. not necessarily x and y).
    \end{itemize}
    
\section{Domain and Co-Domain}
\begin{itemize}[noitemsep]
   \item The domain of the Beta function depends on the limits of the integral function, having a higher limit as well as a lower limit during which the required output of a given function can be obtained.
   \item The co-domain of a function depends on the domains. Here we have to manipulate the co-domains in the predefined form by solving the given problem and converting it into a beta function which can be executed only in some particular domain. Various examples of co-domains are:
    \begin{itemize}
        \item $B(x,y)$ = $\int_{0}^{\frac{\pi}{2}} (\sin \theta)^{2x-1} (\cos \theta)^{2x-1}$
        \item $B(x,y)$ = $\int_{0}^{\infty} \frac{t^{x-1}}{(1+t)^{x+y}} dt$
    \end{itemize}
    \end{itemize}
\begin{thebibliography}{9}
\bibitem{wiki}
% https://en.wikipedia.org/wiki/Beta_function
\url{https://en.wikipedia.org/wiki/Beta\_function}
\end{thebibliography}
    
    
\end{document}
